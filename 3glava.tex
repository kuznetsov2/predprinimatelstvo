\section{Основные причины сопротивления изменениям}

Внедрение управленческих изменений неизменно вызывает сопротивление сотрудников. Это сопротивление существенно затрудняет проведение каких-либо преобразований и при отсутствии специальных мер по управлению сопротивлением и преодолению его последствий может свести к нулю усилия по проведению изменений.

Под сопротивлением персонала изменениям системы управления принято понимать негативную реакцию системы, групп и отдельных лиц, затрудняющую процесс проведения изменений, угрожающую культуре организации и структуре власти. Сопротивление — это первая реакция на изменения, так как людям требуется время, чтобы оценить издержки и выгоды перемен для себя.

Независимо от природы изменения работники стремятся защититься от его последствий, используя жалобы, проволочки, пассивное сопротивление, которые могут перерасти в саботаж и падение интенсивности труда.

По мнению зарубежных специалистов, основная причина сопротивления изменениям заключается в том, что люди воспринимают значительные изменения как разрушение своих ожиданий относительно будущего, как потерю контроля над ситуацией.

Сопротивление изменениям происходит и просто от сознания того, что все они что-то нарушают. Психологической основой сопротивления являются привычки и инерция, страх перед неизвестным.
Людям трудно отказаться от старых привычек и учиться новому.
Тем более что при всяких изменениях создается угроза изменения статуса индивидов, угроза влиятельным формальным и неформальным группам, а нередко и перспективам деятельности всей организации.

М.Х. Мескон, М. Альберт, Ф. Хедоури выявили следующие причины, вызывающие индивидуальное сопротивление менеджеров изменениям системы управления:
\begin{itemize}
	\item нехватка профессиональных знаний и навыков;
\item ощущение потерь (материальных ресурсов, власти, привычных методов работы);
\item внедрение новых формальных процедур;
\item несоответствие ценностей работника корпоративной системе ценностей;
\item перестановки в структуре власти;
\item убежденность, что изменения ничего хорошего не принесут;
\item нехватка времени на решение стратегических вопросов;
\item нехватка ресурсов;
\item неопределенность вследствие нехватки информации;
\item необходимость деятельности, не отвечающей характеру, темпераменту.
\end{itemize}

Однако основной причиной сопротивления, по мнению авторов данного исследования, являются неизбежные изменения культуры организации. Далее следует вывод, что существует возможность управления процессом сопротивления через управление организационной культурой.

Сопротивление может быть связано не с самими изменениями, а с методами их проведения, когда сотрудники недовольны ограничениями в информации, не приемлют авторитарного подхода, не предполагающего их участия в осуществлении перемен. Поводом к сопротивлению нередко становится ощущение сотрудниками несправедливости, вызванное тем, что выгоды проводимых ими изменений присваиваются кем-то другим.

Многие люди в организации не включаются в изменения до тех пор, пока они реально не поймут, в чем заключается их задача, почему ее надо выполнять и какой выигрыш они получат, работая по новым правилам. Причиной сопротивления в этом случае становится недостаток информации о проводимых в организации изменениях.

По мнению Б.Н. Герасимова, эгоистичность поведения является не причиной, а следствием недостаточной ориентированности развития предприятия на социальные интересы и усиление трудовой мотивации. В подтверждение мнения о внешней по отношению к отдельному индивиду природе значительного сопротивления изменениям в организации необходимо подчеркнуть наличие тесной взаимосвязи между отношением персонала к изменениям и его культурой.

Исследования американских ученых показали, что добившиеся значительных успехов в обновлении всех аспектов деятельности компании создали организационную культуру, способствующую изменениям и включающую в себя следующие элементы: дисциплину, поддержку и доверие. Дисциплина способствует выполнению служащими своих обязанностей без принуждения. Для системы поддержки характерны обучение, помощь и руководство. Доверие проявляется в прозрачных, открытых процессах управления.

По мнению И.В. Шекшни, российская национальная культура вообще и организационная культура в частности не предрасположены к делегированию. Руководители отечественных промышленных предприятий готовы делегировать ответственность и редко способны делегировать право принимать решения. Таким образом, реорганизация в условиях отечественной иерархической культуры, как правило, идет только сверху. В таких условиях о широком информировании работников, равенстве и соучастии в реформировании системы управления не может быть и речи.

Характерной особенностью людей, живущих в России, входящих или не входящих в какие бы то ни было организации, является то, что они воспринимают среду своей жизнедеятельности как враждебную, несправедливую и пугающе неопределенную. Следствия этого таковы:
\begin{itemize}
	\item если официальные правила и законы несправедливы, то особое значение придается неофициальным, неписаным правилам;
\item происходит поиск и создание закрытой группы людей, связанных взаимным доверием и собственными правилами.
\end{itemize}

Организационная культура на отечественных предприятиях часто строится по подобному принципу. Опасность заключается в том, что такие системы трудно меняют имеющиеся приемы и способы существования в результате изменений внешней среды.

Сопротивление персонала организации проводимым изменениям подразделяется на индивидуальное и групповое.

Причинами индивидуального сопротивления являются: эгоистический интерес, страх потерять положение, власть, наличие неформальных связей; неправильное понимание изменений и недостаток доверия к лицам, их осуществляющим; низкий уровень готовности к изменениям; различные оценки необходимости и последствий изменений.

Групповое сопротивление возникает из-за того, что предприятие является политической системой, в которой существуют группы с различными «политическими» интересами. Реакция этих групп может зависеть от того, как, по их мнению, проводимые изменения повлияют на баланс сил.

Б.З. Мильнер подчеркивает, что кроме страха личных утрат, отсутствия понимания и доверия, неопределенности вследствие недостатка информации, противодействие изменениям может быть вызвано тем, что работники организации оценивают текущую ситуацию иначе, чем проводники инноваций. Менеджеры различных отделов стремятся к разным целям, а внедрение инноваций зачастую приуменьшает значение достижений некоторых из них.

А.И. Пригожин отмечает, что сопротивление изменениям возникает и начинает усиливаться через один-два месяца после начала преобразований, это происходит по следующим причинам:
\begin{itemize}
	\item увеличивается нагрузка на управленческую команду и большую часть персонала, что предполагает дополнительный объем работ;
\item начинают проявляться латентные или сознательно скрываемые проблемы организации, на руководителя обрушивается лавина новых и обострившихся старых проблем;
\item в период активизации развития отчетливо выявляется соответствие или несоответствие персонала занимаемым должностям, часто обнаруживаются новые лидеры, обостряется борьба позиционных групп за сферы влияния.
\end{itemize}

Принимая во внимание существование индивидуального и группового сопротивления изменениям в организации, можно охарактеризовать различия в причинах возникновения этих видов сопротивления следующим образом:
\begin{itemize}
	\item индивидуальное сопротивление в целом объясняется негативной реакцией отдельных работников, вызванной восприятием значительных изменений как нарушения своих планов относительно будущего, неуверенностью в собственной компетентности и возможности успешного завершения преобразований, а также отсутствием «привычки к изменениям»;
\item групповое (системное, организационное) сопротивление возникает из-за несоответствия системы ценностей организации поставленным целям изменений, значительной величины культурного разрыва, несоответствия управленческого потенциала реализуемой стратегии преобразований.
\end{itemize}


















